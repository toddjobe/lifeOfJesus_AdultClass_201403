\lecture{The Poor in Spirit Need Jesus}{March 02}
\begin{frame}

\frametitle{What are you really saying when \dots}
\begin{columns}
\column{4.5cm}
you complain about\\things at church,\\~\\

you think others are\\less spiritual than you,\\~\\

you cannot let go\\after someone hurts you.\\~\\ 

you are more merciful to\\yourself than to others\dots\\~\\~\\
\centering\uncover<2->{\alert{``I deserve to be here.''}}
\column{5.5cm}
\uncover<2->{\includegraphics[width=\textwidth]{graphics/entitlement.jpg}}
\end{columns}
~\\
\end{frame}


\begin{frame}
\frametitle{This week's readings remind us that\\we are all sinners.}

Peter recognizes his sinfulness\\~\\
Jesus forgives the sins of a paralytic man.\\~\\
Jesus call Matthew, a sinner who wants to repent.


\end{frame}

\begin{frame}
\frametitle{When we remember our sinfulness,\\we realize  our dependence on Jesus.}
\begin{verse}
Blessed are the poor in spirit,\\for theirs is the kingdom of heaven.

Blessed are those who mourn,\\for they shall be comforted.

Blessed are the meek,\\for they shall inherit the earth.

\attrib{\bibleverse{Matthew}(5:3-5)}
\end{verse}
\end{frame}

\begin{frame}
\frametitle{\insertlecture}
\tableofcontents[sectionstyle=show/show]
\end{frame}

\section{We do not deserve to be in the presence of Jesus.}
\begin{frame}
\frametitle{Peter knows what kind of person he is.}
\framesubtitle{\bibleverse{Luke}(5:1-11)}

\begin{columns}
\column{4.5cm}
He hears Jesus teach\\of the kingdom.\\~\\
He sees Jesus perform a personally relevant miracle.\\~\\
The miracle produces belief --- and guilt.\\~\\
\column{6cm}
\includegraphics[width=\textwidth]{graphics/prophet-isaiah-1968.jpg}\\
\tiny{A similar event happened to Isaiah.}\\~\\
%\raggedleft\tiny{Prophet Isaiah, Marc Chagall, 1968}
\end{columns}
\begin{quote}
When Simon Peter saw what had happened,\\he bowed down before Jesus and said,\\``Go away from me, Lord. I am a sinful man!''

\attrib{\bibleverse{Luke}(5:8), NCV}
\end{quote}
\end{frame}

\begin{frame}
\frametitle{The gospel is a better mirror than a bat.}
\begin{columns}
\column{5cm}
Do not make the sins of others\\too big.\\~\\
Do not make your own sins\\too small.\\~\\~\\
\column{5cm}
\includegraphics[width=\textwidth]{graphics/beam.jpg}\\
\end{columns}
~\\Peter probably was a horrible person.\\
But, Jesus could transform him.
\end{frame}

\section{But, Jesus is the only one with power to forgive us.}
\begin{frame}
\frametitle{Jesus heals a paralytic physically and spiritually.}
\framesubtitle{\bibleverse{Mark}(2:1-12)}
\begin{columns}
\column{5.5cm}
The paralytic has some\\unashamed friends.\\~\\
It is `their' faith that\\prompts the Lord to forgive.\\~\\
\column{4.5cm}
\includegraphics[width=\textwidth]{graphics/paralytic.jpg}
\end{columns}
\centering~\\
Jesus has the power to grant \emph{everything} man needs.
\end{frame}

\begin{frame}
\frametitle{Where else can you go for forgiveness but to Jesus?}
\begin{quote} 
And when Jesus saw \textbf{their faith}, he said to the paralytic, ``Son, \textbf{your sins are forgiven}.''

\attrib{\bibleverse{Mark}(2:5)}
\end{quote}

\begin{quote}
Is anyone among you sick? Let him call for the elders of the church, and let them pray over him, anointing him with oil in the name of the Lord. And \textbf{the prayer of faith will save the one who is sick, and the Lord will raise him up. And if he has committed sins, he will be forgiven}. Therefore, confess your sins to one another and pray for one another, that you may be healed. The prayer of a righteous person has great power as it is working.

\attrib{\bibleverse{James}(5:14-16)}
\end{quote}

\end{frame}

\section{So, He comes to those who want forgiveness.}

\begin{frame}
\frametitle{The healthy don't need a doctor, but the sick do.}
\framesubtitle{\bibleverse{Luke}(5:27-39)}
Jesus befriends Matthew and his cohort of tax collectors.\\~\\
Jesus wants those \emph{sinners} to become \emph{repenters}.\\~\\
The Pharisees have a totally different world-view\\where the righteous and the sinner never change.
\end{frame}

\begin{frame}
\frametitle{Christians are defined by their willingness to change.}
Old wineskins are tough and unyielding.\\
New wineskins are receptive to God's teachings.\\
Christians must \emph{live} as new wineskins.\\~\\~\\
This principle can be applied every time we come together.
\begin{itemize}
\item Study is not just about confirmation. It's about learning.
\item Resist the urge to shoot someone's thought down immediately.
\item Every generation has to interpret the Bible for themselves.
\end{itemize}

\end{frame}

\section*{Conclusion}
\begin{frame}
\frametitle{}
Know who you are\ldots\\
A sinner, forgiven.\\~\\
Be willing to change\ldots\\
by Jesus' power.

\end{frame}

\section*{Tough Passages}

\begin{frame}
\begin{quote}
\dots And the power of the Lord was with him to heal \dots

\attrib{\bibleverse{Luke}(5:17b)}
\end{quote}

\end{frame}
