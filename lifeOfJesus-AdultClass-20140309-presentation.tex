%!TEX root = lifeOfJesus_AdultClass_201403_presentation.tex

\begin{frame}
\frametitle{Jesus' attitude toward the Law}
This week's readings focus on Jesus and the Sabbath.\\~\\
In them we learn about 
\begin{itemize}
\item Jesus' authority 
\item His merciful application of the Law
\item His common sense approach to teaching the Law.
\end{itemize}
~\\~\\
But first, some history \dots\\

\end{frame}

\begin{frame}
\frametitle{The Sabbath law was pretty simple}
\begin{quote}
``Six days work shall be done, but on the seventh day you shall have a Sabbath of solemn rest, holy to the Lord. \textbf{Whoever does any work on it shall be put to death}. You shall kindle no fire in all your dwelling places on the Sabbath day.''

\attrib{\bibleverse{Exodus}(35:2-3)}
\end{quote}

\begin{quote}
``\dots but the seventh day is a Sabbath to the Lord your God. \textbf{On it you shall not do any work}, you or your son or your daughter or your male servant or your female servant, or your ox or your donkey or any of your livestock, or the sojourner who is within your gates, that your male servant and your female servant may rest as well as you. You shall remember that you were a slave in the land of Egypt, and the Lord your God brought you out from there with a mighty hand and an outstretched arm.''

\attrib{\bibleverse{Deuteronomy}(5:13-15a)}
\end{quote}
\end{frame}

\begin{frame}
\frametitle{God took the Sabbath law seriously}
\begin{quote}
While the people of Israel were in the wilderness, they found a man gathering sticks on the Sabbath day. And those who found him gathering sticks brought him to Moses and Aaron and to all the congregation. 34 They put him in custody, because it had not been made clear what should be done to him. And \textbf{the Lord said to Moses, ``The man shall be put to death}; all the congregation shall stone him with stones outside the camp.'' 36 And all the congregation brought him outside the camp and stoned him to death with stones, as the Lord commanded Moses.

\attrib{\bibleverse{Numbers}(15:32-36)}
\end{quote}

\end{frame}

\begin{frame}
\frametitle{39 Activities Defined as Work by the Mishna}
\begin{multicols}{3}
\begin{itemize}
\item Planting
\item Plowing
\item \textbf{Reaping}
\item Gathering
\item Threshing
\item Winnowing
\item Sorting
\item Grinding
\item Sifting
\item Kneading
\item Cooking
\item Shearing
\item Laundering
\item Combing wool
\item Dyeing
\item Spinning
\item Warping
\item Threading
\item Weaving
\item Separating
\item Tying
\item Untying
\item Sewing
\item Tearing
\item Trapping
\item Slaughtering
\item Skinning
\item Preserving
\item Smoothing
\item Scoring
\item Cutting
\item Writing
\item Erasing
\item Building
\item Demolition
\item Extinguishing
\item Igniting
\item Finishing
\item \textbf{Carrying}
\end{itemize}
\end{multicols}
\end{frame}

\begin{frame}
\frametitle{Who cares about the Sabbath?}
It was clearly important to God.\\
It was not a very specific command.\\
It caused the Jews to treat others poorly.\\~\\
\emph{Are there commands like that in the New Testament?}\\
\emph{What should my attitude be toward such commands?}\\
\end{frame}

\begin{frame}
\frametitle{\insertlecture}
\tableofcontents[sectionstyle=show/show]
\end{frame}

\section{Jesus is diety}
\begin{frame}
\frametitle{Jesus' works proclaim His Lordship}
\framesubtitle{\bibleverse{John}(5:1-18)}
Jesus healed a lame man on the Sabbath\\in front of many people.\\~\\
The Jews knew that He was\\claiming equal authority with God.\\~\\
This made the Jews\\want to kill him
\end{frame}

\begin{frame}
\frametitle{Jesus unmistakeably says He is from God}
\framesubtitle{\bibleverse{John}(5:19-47)}
\textbf{Jesus has authority}\\
The Son does what He sees the Father doing.\\
The Father gives life, and the Son gives life.\\
The Father has given the power of judgement to the Son.\\~\\
\textbf{Jesus has witnesses}\\
John the Baptist\\
Jesus' works\\
The Father through the Scripture\\~\\
\begin{quote}
``You search the Scriptures because you think that in them you have eternal life; and it is they that bear witness about me''

\attrib{\bibleverse{John}(5:39)}
\end{quote}
\end{frame}

\section{Jesus is merciful}

\begin{frame}
\frametitle{Jesus is Lord of the Sabbath}
\framesubtitle{\bibleverse{Matthew}(12:1-8)}
The disciples pick grain on the Sabbath,\\a clear (if small) violation of the Mishna.\\~\\
But, is it work?\\~\\~\\
Jesus proves He is Lord\\not because He \emph{changes} the Law,\\but because He \emph{interprets} it.\\~\\
\centering
\begin{quote}
``Do not think that I have come to abolish the Law or the Prophets; I have not come to abolish them but to fulfill them.''

\attrib{\bibleverse{Matthew}(5:17)}
\end{quote}
\end{frame}

\begin{frame}
\frametitle{Why was David not punished for eating the priests' bread?}
Jesus says that it was unlawful.\\
But, God is merciful.\\~\\~\\
\begin{center}
Think about how many sins Jesus saw every day.
\end{center}
\end{frame}

\section{Jesus appeals to reason}

\begin{frame}
\frametitle{Is it lawful to heal on the Sabbath?}
\framesubtitle{\bibleverse{Matthew}(12:9-14)}
Healing didn't fall exactly in the 39 categories of work.\\~\\
Jesus heals a man's withered hand.\\~\\
He appeals to common sense to determine that\\
it was lawful to do good on the Sabbath\\~\\~\\
Correct interpretation of the Bible depends on a proper attitude.\\~\\~\\
\end{frame}

\section*{Conclusion}

\begin{frame}
\frametitle{``Helping'' Christians apply God's commmands.}
These Sabbath traditions were originally designed to\\
help people make application of God's commands.\\~\\
This is a dangerous trap for those of us\\who take God's commands seriously.\\~\\
Example:\\
The New Testament is not very specific ``modest'' dress.\\
So, well-meaning Christians devised a number of convoluted arguments to tell Christians what modest dress looks like using:
\begin{itemize}
\item The definition of the word \emph{tunic}.
\item The garments that preists wore.
\item Garments worn at the time of Abraham.
\end{itemize}

\end{frame}

\begin{frame}
\frametitle{\insertlecture}
\tableofcontents[sectionstyle=show/show]
\end{frame}

\section*{Tough Passages}
\begin{frame}
\frametitle{Some newer translations leave out \bibleverse{John}(5:4)}
\framesubtitle{For example: ESV, NIV}

\footnotesize  
\begin{quote}
\textsuperscript{3} In these lay a great multitude of impotent folk, of blind, halt, withered, \textbf{waiting for the moving of the water. \textsuperscript{4} For an angel went down at a certain season into the pool, and troubled the water: whosoever then first after the troubling of the water stepped in was made whole of whatsoever disease he had.} \textsuperscript{5} And a certain man was there, which had an infirmity thirty and eight years.

\attrib{\bibleverse{John}(5:3-5), KJV}
\end{quote}
\normalsize 
Newer translations are probably right for the following reasons.
\begin{itemize}
\item KJV is from the \textit{Textus Recepticus} whose compiled manuscripts date from 1100 A.D. or later.
\item No manuscripts prior to 500 A.D. contains verse 4.
\item Multiple Greek manuscripts around 900 A.D. have marks questioning the validity of this verse.
\item The verse has multiple words occurring nowhere else in John.
\item The verse has a large number of variants in manuscripts.
\end{itemize}

\end{frame}

\begin{frame}
\frametitle{Where does ``I desire mercy and not sacrifice'' come from?}
\framesubtitle{\bibleverse{Matthew}(1:6)}
\begin{verse}
For I desire \emph{steadfast love} and not sacrifice,\\
the knowledge of God rather than burnt offerings.\\
\attrib{\bibleverse{Hosea}(6:6),ESV}
\end{verse}~\\
Jesus quotes the LXX, which says \emph{mercy},\\
But, the earliest manuscripts say \emph{steadfast love}.\\~\\
Think about how Hosea had to show mercy;\\
by taking back a cheating wife.
\end{frame}

%\begin{frame}
%\frametitle{The Priests Profane the Sabbath}

%\begin{quote}
%``On the Sabbath day, two male lambs a year old without blemish, and two tenths of an ephah of fine flour for a grain offering, mixed with oil, and its drink offering: 10 this is the burnt offering of every Sabbath, besides the regular burnt offering and its drink offering.''

%\attrib{\bibleverse{Numbers}(28:9-10)}
%\end{quote}

%\end{frame}

%\begin{frame}
%\frametitle{Sometimes we do nothing\\because we're afraid of doing something wrong}
%There is nothing wrong with avoiding temptation.
%``I don't give to orphans homes\\because they are all run by churches.''\\
%\end{frame}

