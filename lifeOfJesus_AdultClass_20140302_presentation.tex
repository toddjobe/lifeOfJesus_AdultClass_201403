\documentclass{beamer}

\usepackage{lmodern}
\usepackage{attrib}
\usepackage{verse}
\usepackage{altverse}
\usepackage{bibleref}

%\graphicspath{graphics/}

\title{Having a Heart Like Jesus}
%\subtitle{The Life Of Jesus}
%\author[WSCOC]{Walnut Street Church of Christ}
\date{March, 2014}

\AtBeginSection[]
{
\begin{frame}
\frametitle{\insertlecture}
\tableofcontents[currentsection]
\end{frame}
}

\includeonlylecture{March 09}

\begin{document}
\frame{\titlepage}


\begin{frame}
\frametitle{We need consistent attitude adjustments.}
Seven of the nine beatitudes focus on the condition of our heart.\\~\\
\begin{quote}
For the word of God is living and active, sharper than any two-edged sword, piercing to the division of soul and of spirit, of joints and of marrow, and \textbf{discerning the thoughts and intentions of the heart}.

\attrib{\bibleverse{Heb}(4:12), ESV}
\end{quote}

\begin{quote}
What I am eager for is that all the Christians there will be filled with \textbf{love that comes from pure hearts},and that their minds will be clean and their faith strong.

\attrib{\bibleverse{ITim}(1:5), TLB}
\end{quote}
\end{frame}
\lecture{Attitudes of the Poor in Spirit}{March 02}
\begin{frame}

\frametitle{What are you really saying when \dots}

you are a snob,

you whine about things at church,

you are more merciful to yourself than to others\dots\\~\\

\centering\uncover<2->{\alert{``I deserve to be here.''}}

\end{frame}


\begin{frame}
\frametitle{This week's readings dovetail nicely with some beatitudes.}

Peter says to the Lord,\\``Depart from me for I am a sinful man.''\\~\\
Jesus forgives the sins of a paralytic man.\\~\\
Jesus calls Matthew, a tax collector.

\end{frame}

\begin{frame}
\frametitle{We need a humble heart that recognizes\\our dependence on Jesus.}
\end{frame}

\begin{frame}
\frametitle{\insertlecture}
\tableofcontents[sectionstyle=show/show]
\end{frame}

\section{We do not deserve to be in the presence of Jesus.}
\begin{frame}
\frametitle{Peter knows what kind of person he is.}
\framesubtitle{\bibleverse{Luke}(5:1-11)}

\begin{columns}
\column{4.5cm}
He hears Jesus teach\\of the kingdom.\\~\\
He sees Jesus perform a personally relevant miracle.\\~\\
The miracle produces belief --- and guilt.\\~\\
\column{6cm}
\includegraphics[width=\textwidth]{graphics/prophet-isaiah-1968.jpg}\\
\tiny{A similar event happened to Isaiah.}\\~\\
%\raggedleft\tiny{Prophet Isaiah, Marc Chagall, 1968}
\end{columns}
\begin{quote}
When Simon Peter saw what had happened,\\he bowed down before Jesus and said,\\``Go away from me, Lord. I am a sinful man!''

\attrib{\bibleverse{Luke}(5:8), NCV}
\end{quote}
\end{frame}

\begin{frame}
\frametitle{The gospel is a great mirror.}
\begin{columns}
\column{5cm}
Do not make the sins of others\\too big.\\~\\
Do not make your own sins\\too small.\\~\\~\\
\column{5cm}
\includegraphics[width=\textwidth]{graphics/beam.jpg}\\
\end{columns}
~\\Peter probably was a horrible person.\\
But, Jesus could transform him.
\end{frame}

\section{But, Jesus is the only one with power to forgive us.}
\begin{frame}
\frametitle{Jesus heals a paralytic physically and spiritually.}
\framesubtitle{\bibleverse{Mark}(2:1-12)}
\begin{columns}
\column{5cm}
The paralytic has some unashamed friends.\\~\\
It is `their' faith that prompts the Lord to forgive.\\~\\
\column{5cm}
\includegraphics[width=\textwidth]{graphics/paralytic.jpg}
\end{columns}
\centering~\\
Jesus has the power to grant \emph{everything} man needs.
\end{frame}

\begin{frame}
\frametitle{Where else can you go for forgiveness but to Jesus?}
\begin{quote} 
And when Jesus saw \textbf{their faith}, he said to the paralytic, ``Son, \textbf{your sins are forgiven}.''

\attrib{\bibleverse{Mark}(2:5)}
\end{quote}

\begin{quote}
Is anyone among you sick? Let him call for the elders of the church, and let them pray over him, anointing him with oil in the name of the Lord. And \textbf{the prayer of faith will save the one who is sick, and the Lord will raise him up. And if he has committed sins, he will be forgiven}. Therefore, confess your sins to one another and pray for one another, that you may be healed. The prayer of a righteous person has great power as it is working.

\attrib{\bibleverse{James}(5:14-16)}
\end{quote}

\end{frame}

\section{So, He comes to us, when we could not come to Him.}

\begin{frame}
\frametitle{The healthy don't need a doctor, but the sick do.}
\framesubtitle{\bibleverse{Luke}(5:27-39)}
Jesus befriends Matthew and his cohort of tax collectors.\\
Jesus wants those \emph{sinners} to become \emph{repenters}.\\~\\
The Pharisees have a totally different world-view\\where the righteous and the sinner never change.
\end{frame}

\begin{frame}
\frametitle{Christians are defined by their willingness to change.}
Old wineskins are tough and unyielding.\\
New wineskins are receptive to God's teachings.\\
Christians must \emph{live} as new wineskins.\\~\\~\\
This principle can be applied every time we come together.
\begin{itemize}
\item Study is not just about confirmation. It's about learning.
\item Resist the urge to shoot someone's thought down immediately.
\item Every generation has to interpret the Bible for themselves.
\end{itemize}

\end{frame}

\section*{Conclusion}
\begin{frame}
\frametitle{Know who you are.}
A sinner,~~~~~forgiven.
\end{frame}

\section*{Tough Passages}

\begin{frame}
\begin{quote}
\dots And the power of the Lord was with him to heal \dots

\attrib{\bibleverse{Luke}(5:17b)}
\end{quote}

\end{frame}

\lecture{I Require Mercy and Not Sacrifice}{March 09}

\begin{frame}
\frametitle{Attention Getter}
something
\end{frame}

\begin{frame}
\frametitle{Need}
some bible verses
\end{frame}

\begin{frame}
\frametitle{task}
current reading
\end{frame}

\begin{frame}
\frametitle{Solution in short}
\end{frame}

\begin{frame}
\frametitle{\insertlecture}
\tableofcontents[sectionstyle=show/show]
\end{frame}

\section{The Sabbath was for man's good}
\begin{frame}
\frametitle{Man healed at pools of Bethesda}
Man is healed at the pools of Bethesda 
\end{frame}

\begin{frame}
\frametitle{}
\end{frame}

\section{Jesus is Lord of the Sabbath}

\begin{frame}
\frametitle{Disciples pick grain on the Sabbath}
\end{frame}

\begin{frame}
\frametitle{Situation Ethics}
\end{frame}

\section{God's law does good.}

\begin{frame}
\frametitle{Man's hand healed on the Sabbath}
\end{frame}

\begin{frame}
\frametitle{Sometimes we do nothing because we're afraid of doing something wrong}
\end{frame}

\section*{Conclusion}
\begin{frame}
\frametitle{Be merciful}
\end{frame}

\section*{Tough Passages}
\begin{frame}
\frametitle{Some newer translations leave out \bibleverse{John}(5:4)}
\framesubtitle{For example: ESV, NIV}

\footnotesize  
\begin{quote}
\textsuperscript{3} In these lay a great multitude of impotent folk, of blind, halt, withered, \textbf{waiting for the moving of the water. \textsuperscript{4} For an angel went down at a certain season into the pool, and troubled the water: whosoever then first after the troubling of the water stepped in was made whole of whatsoever disease he had.} \textsuperscript{5} And a certain man was there, which had an infirmity thirty and eight years.

\attrib{\bibleverse{John}(5:3-5), KJV}
\end{quote}
\normalsize 
Newer translations are probably right for the following reasons.
\begin{itemize}
\item KJV is from the \textit{Textus Recepticus} whose compiled manuscripts date from 1100 A.D. or later.
\item No manuscripts prior to 500 A.D. contains verse 4.
\item Multiple Greek manuscripts around 900 A.D. have marks questioning the validity of this verse.
\item The verse has multiple words occurring nowhere else in John.
\item The verse has a large number of variants in manuscripts.
\end{itemize}

\end{frame}

\end{document}
