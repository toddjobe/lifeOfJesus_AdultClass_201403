\documentclass{tufte-handout}

%\geometry{showframe}% for debugging purposes -- displays the margins

\usepackage{amsmath}

% Set up the images/graphics package
\usepackage{graphicx}
\setkeys{Gin}{width=\linewidth,totalheight=\textheight,keepaspectratio}
\graphicspath{{graphics/}}

\title{The Life of Jesus}
\date{March 2014}  % if the \date{} command is left out, the current date will be used

% The following package makes prettier tables.  We're all about the bling!
\usepackage{booktabs}

% The units package provides nice, non-stacked fractions and better spacing
% for units.
\usepackage{units}

% The fancyvrb package lets us customize the formatting of verbatim
% environments.  We use a slightly smaller font.
\usepackage{fancyvrb}
\fvset{fontsize=\normalsize}

% Small sections of multiple columns
\usepackage{multicol}

% Provides paragraphs of dummy text
\usepackage{lipsum}

% These commands are used to pretty-print LaTeX commands
\newcommand{\doccmd}[1]{\texttt{\textbackslash#1}}% command name -- adds backslash automatically
\newcommand{\docopt}[1]{\ensuremath{\langle}\textrm{\textit{#1}}\ensuremath{\rangle}}% optional command argument
\newcommand{\docarg}[1]{\textrm{\textit{#1}}}% (required) command argument
\newenvironment{docspec}{\begin{quote}\noindent}{\end{quote}}% command specification environment
\newcommand{\docenv}[1]{\textsf{#1}}% environment name
\newcommand{\docpkg}[1]{\texttt{#1}}% package name
\newcommand{\doccls}[1]{\texttt{#1}}% document class name
\newcommand{\docclsopt}[1]{\texttt{#1}}% document class option name

\begin{document}

\maketitle% this prints the handout title, author, and date

\begin{abstract}
The main theme is attitude
1) Sick need a Physician
2) I desire mercy and not sacrifice
3) Sermon on the Mount Chapter 5
4) Sermon on the Mount Chapter 6
5) Sermon on the Mount Chapter 7 - summary 

Things left to do:
1) Get the per week text ready.
2) Get the powerpoints ready
3) Get the text ready.


Here is the text for the abstract.

Lesson on the Sabbath - that's comparing the old law and the new.  Or how the Pharisees and the 

I'd like to think that I take obeying God's law seriously.  We have a whole set of epistles whos main focus is to help Christians live holy lives.  Jesus tells us what it means to be holy. 

Maybe we'll do the events on chronological order, and just have 3 themes.
Possible themes:

Jesus teaches us how to have a heart like His.

1) Sabbath (Old law vs. New)
2) Having the right heart (the beatitudes)
3) authority of Jesus
4) mercy and not sacrifice

jesus Preaches on Simon's boat - nothing

Miraculous Catch of Fish - Seeing yourself for what you are.

Jesus heals a leper - don't tell anyone...who knows what this means.

Jesus cures a paralytic - "The power of the Lord was with Him to heal" - I don't know what this means. - Jesus has the authority to forgive sins.  Love of the friends.  They take healing seriously.  We should take salvation just as seriously.  Think outside the box to make sure we have access to Jesus.

Levi called to be a disciple - he's a tax gatherer

Reception at Levi's house - 1) Why do you eat with sinners?  It's not the well who need a physician, but the sick.  Mercy and not sacrifice.  Old wineskins, new wineskins

Parables at Levi's receiption - fasting/bridegroom, oldwineskins, new wineskins

Jesus in Jerusalem for 2nd passover - nothing

Man healed at pools of bethesda - nothing

Jesus challenged for healing on the Sabbath - people try to get in the way of doing right.  Jesus has authority.

Disciples pick grain on the Sabbath - David and the priests profane the Sabbath.  They are contrasting.  David clearly broken the law.  Priests profane the Sabbath, but are guiltless. Mercy and Not sacrifice again.

Man's hand healed on the sabbath - Save life or destroy it.

Jesus withdraws to the sea - this actually has a lot of good stuff in it about why Jesus did not have confrontations at the time with Pharisees.  He is actively seeming weak.  He doesn't quarrel or cry aloud in the streets.  

Many follow Jesus to be healed - teaching in synagogues, proclaiming the gospel, healing diseases.

Jesus prays on the mountain - prayed all night in seeming preparation for selecting the apostles

Twelve apostles selected - nothing

Jesus descends to heal the multitude - nothing

Jesus ascends to heal the multitude - This is WRONG one is the 'sermon on the plain'.  The other is the 'sermon on the mount'.  I think Jesus just gave the same sermon everywhere he went.

Sermon on the mount - differences between luke and matthew.  Obviously, matthew is longer.  Luke includes woes to go along with the beatitudes.  36 be merciful as your heavenly father is merciful.

\end{abstract}

%\printclassoptions
introduction 

\section{Lesson 1}\label{sec:lesson1}
here is the text for lesson 1



\section{Lesson 2}\label{sec:lesson2}
here is the text for lesson 2.

\section{Lesson 3}\label{sec:lesson3}
Here is the text for lesson 3.

\section{Lesson 4}\label{sec:lesson4}
Here is the text for lesson 4.

\section{Lesson 5}\label{sec:lesson5}
Here is the text for lesson 5.

Probably will be the Sermon on the Mount.
Summary of the sermon on the mount.
The beatitudes.
let your light shine before men.
...but I say unto you.
good works from the heart not to be seen by men
Judge not that you be not judged
False prophets and bearing fruit


\end{document}
