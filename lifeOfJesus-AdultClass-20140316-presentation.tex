%!TEX root = lifeOfJesus_AdultClass_201403_presentation.tex
\note{
Summary of Chapters 5 - 6 
If you're heart is not right, then you're not right.

You should not teach the sermon on the mount.  You should simply read it and let the lessons sink in.

Should I hit the highlights, hit the verses we get wrong or do a summary?

Really what I would like for people to learn from class is to meditate on how to do these things.  It's funny, we come up with all sorts of ways to accomplish goals as if the Bible is sort of incomplete when it comes to doing things.  And, in a way, we have to figure out what areas of our life apply to the lessons we're learning.
Some places the sermon on the mount tells you just how to do stuff.  In other places it's a little more vague.


5:13-16 Salt and Light.
5:17-20 This is the way that God has always wanted us to be.
5:21-32 When applying the Lord's commands, you have to think about the state of the heart that God is trying to get you to have.  Let's have some other examples.
5:38-end We have a different standard of living.
6:1-18 Down to earth person.  Your outside and your inside match. 
6:19-34 Priorities

The Beatitudes really contrast with how you might try to find happiness if you weren't taught by Jesus
How to be happy

They surround themselves with happy people
The cultivate resilience
They try to be happy
They are mindful of the good
They appreciate simple pleasures
They devote some of their time to giving
They let themselves lose track of time
They nix the small talk for deeper conversation
They make a point to listen
They uphold in-person connections
They look on the bright side

1) Be Proactive
2) Begin With The End in Mind
3) Put First Things First
4) Think Win/WIn
5) Seek First to Understand, THen to be Understood
6) Synergize
7) Sharpen the Saw

You have to visualize what you want

You don't need to interpret it.  You just have to find ways to incorporate it.  You have to ask, ``Where does this apply to my life?'' 

Heaven is only for people that know they don't belong there.
grieve, not only for sin but just in general for the things that get you down in life, 
You sort of have to lose you're own identity to let God give you His.


What is it about the things you get?
I think that you can't really take them independently.
I think there's a tendency to think about.  There are some Christians that are this.  And others that are good at meekness, and others that are good at being peacable.  But, they aren't independent.  These describe a whole person.
And to some degree these 
}

\begin{frame}
\frametitle{Background on Sermon the Mount}
This is \emph{the} sermon Jesus taught: on the plain, on the mountain, everywhere.
Jesus taught with authority
\end{frame}

\begin{frame}
\frametitle{What we hope to learn}
You have to look beyond the command to the condition of the heart that allows you to keep the command.
If you're heart is not right, then you're wrong.
\end{frame}

\begin{frame}
\frametitle{\insertlecture}
\tableofcontents[sectionstyle=show/show]
\end{frame}

\section{The Beatitudes}
\begin{frame}
\frametitle{The Beatitudes Discussion}
Poor in Spirit - 
We all know that happy = blessed.  But perhaps we should think of the word fortunate.  He's the luckiest guy on the block. 
If you ave a humble 
\end{frame}

\begin{frame}
\frametitle{The Beatitudes Application}
\end{frame}

\section{Kingdom Living}

\begin{frame}
\frametitle{The kingdom calls for a higher standard of living}
A standard where you make sure that your outward behavior is driven by an inward resolve.
If someone is mad at you, you take the initiative to fix it.
\end{frame}

\begin{frame}
\frametitle{Why was David not punished for eating the priests' bread?}
\end{frame}
